\section{Introduction to ``Emergency Light Driver''}
``Emergency Light Driver'' is~a device for controlling lights used for emergency vehicles like
ambulance, fire truck, road assistance, etc. Also, supports front and rear lights.

Driver can be applied to different vehicles types -- it works the same for different light colors. Although, lights color should be established on vehicle assembly state (different color for road assistance
different for ambulance).

Driver should allow running subsets of lights in few different modes.

%% ------------------------------------------------------------------------------------------ %%

\subsection{Lights sets and modes}
Short summary of sets with available modes.

\begin{enumerate}
	\item Front and rear lights -- Normal white and red lights, present in every vehicle
	\begin{itemize}
		\item Off -- No lights.
		\item Vehicle goes forward -- front white light, rear red light.
		\item Vehicle goes backward -- front red light, rear withe light.
		\item All on -- all white and red are on.
	\end{itemize}
	
	\item Top set
	\begin{itemize}
		\item Off -- No lights.
		\item Flashing alternately -- Few short signals on left side, then the same on right side.
		\item Light alternately -- Constant signal on left side, then the same on right side. Working a bit slower than ``Flashing alternately'' mode.
		\item All on -- all lights are on, no blinking.
	\end{itemize}
		
	\item Contour
	\begin{itemize}
		\item Off - No Lights.
		\item Light alternately - Constant signal on even light then on odd light.
		\item Fase - All lights fades in and out simultaneously.
		\item All on -- all lights are on, no blinking.
	\end{itemize}
\end{enumerate}

%% ------------------------------------------------------------------------------------------ %%

\subsection{Communication}
Driver supports two ways of communication. Communication mode can be selected by setting logic 0~or~1 to dedicated pin.
\newline
Modes:
\begin{itemize}
	\item dedicated input pins for setting wanted light states,
	\item I\textsuperscript{2}C interface\footnote{Or compatible like TWI~\cite{i2c-twi}}
\end{itemize}

%% ------------------------------------------------------------------------------------------ %%

\subsection{Hardware}
Main chip is \textbftt{atmega328}, or compatible. Without quartz oscillator.

Removed oscillator allows building simpler and gives 2 more I/O pins. Without external quartz, chip works with frequency of 8MHz, which is more than enough
even for smooth fading. Connection via I\textsuperscript{2}C with chips with 16MHz oscillator will not be any problem, because I\textsuperscript{2}C works
with a~speed of slowest device~\cite{i2cspeed}.
